\section{ФУНКЦИИ И ГРАФИКИ}
\subsection{Линейная, квадратичная и обратно- пропорциональная функции}


При построении или идентификации графиков следует пользоваться следующим простым правилом. Если  функция задана \newline уравнением $y=f(x)$ и нас интересует, лежит ли некоторая точка $\left(x_0,y_0\right)$ на графике этой функции, то надо просто подставить координаты $\left(x_0,y_0\right)$ в уравнение. Если уравнение удовлетворяется, т.е. выполняется равенство $y_0-f(x_0)$, то точка находится на графике, если же $y_0\neq f(x_0)$, то не находится.

Линейная функция задается выражением $y_0=kx+b$. Если угловой коэффициент $k>0$, функция возрастает, если $k<0$, то убывает. Если , функция постоянна, ее графиком является прямая, параллельная оси $Ox$.

\textbf{1412.}  Функция убывает, варианты 3) и 4) исключаются. Угловой коэффициент в варианте 1) равен $-\frac{1}{3}$. \newline \null \hspace*{\fill} Ответ: $1$.   

\textbf{1413.}  Функция убывает, выбор надо сделать между вариантами 2) и 4). Координаты точки (-2;0), лежащей на графике, удовлетворяют только варианту 4). \newline \null \hspace*{\fill} Ответ: $4$. 

По аналогии решаются задачи \textbf{1414}-\textbf{1419}.

\textbf{1420.} \newline \null \hspace*{\fill} Ответ: А $-$ 2, Б $-$ 3, В $-$ 1.

Приведенные выше соображения позволяют решить задачи \newline \textbf{1421}-\textbf{1427}.

\textbf{1428.}  Функция убывает, поэтому $k<0$. Отрезок, отсекаемый от оси $Oy$, отрицательный, т.е. $b<0$. \newline \null \hspace*{\fill} Ответ: $3$. 

Задачи \textbf{1429}-\textbf{1433}  аналогичные.

\textbf{1434.}\newline \null \hspace*{\fill} Ответ: А $-$ 3, Б $-$ 2, В $-$ 1.

Задачи \textbf{1435}-\textbf{1439}  аналогичные.

\textbf{1440.} \newline \null \hspace*{\fill} Ответ: $-3$.   

\textbf{1441.} \newline \null \hspace*{\fill} Ответ: $2$.  

\textbf{1442.} \newline \null \hspace*{\fill} Ответ: $0$.     

\textbf{1443.} \newline \null \hspace*{\fill} Ответ: $-5$.   

\textbf{1444.} Если точки с координатами $(x_1,y_1)$ и $(x_2,y_2)$ находятся на прямой $y=kx+b$, то угловой коэффициент можно вычислить по формуле:$$k=\frac{y_2-y_1}{x_2-x_1}.$$

Возьмем на графике точки $(0;0)$ и $(2;5)$. Тогда $k=\frac{5}{2}=2,5$. \newline \null \hspace*{\fill} Ответ: $2,5$. 

Аналогично решаются задачи \textbf{1445}-\textbf{1447}.

\textbf{1448.} Ветви параболы направлены вверх: варианты  1) и 4) исключаем. Из графика видно, что вершина находится в точке $(-3;-4)$. Координаты этой точки удовлетворяют уравнению 2) и не удовлетворяют уравнению 3). \newline \null \hspace*{\fill} Ответ: $2$. 

Задачи \textbf{1449},\textbf{1450} решаются аналогично.

\textbf{1451.}  Варианты 1) и 2) исключаем, т.к. ветви параболы направлены вверх. Из вариантов 3) и 4) выбираем 4), т.к. при $x=1-x^2+3x+3=5$, а на графике 3) значение $y$ при $x=1$ отрицательное. \newline \null \hspace*{\fill} Ответ: $4$. 

Задачи \textbf{1452},\textbf{1453} решаются аналогично.

\textbf{1454.} На графике Б) ветви параболы направлены вниз, график соответствует ответу 1). Вершина параболы на графике А) находится в точке $(3;-3)-$ этому графику соответствует уравнение 3). Аналогично убеждаемся, что координаты $(-3;-3)$ вершины параболы В) удовлетворяют уравнению  2). \newline \null \hspace*{\fill} Ответ: А $-$ 3, Б $-$ 1, В $-$ 2.(ответ в задачнике ошибочный!). 

Задачи \textbf{1455}-\textbf{1459} аналогичные.

\textbf{1460.} Ветви параболы направлены вверх, значит $a>0$. Значение функции при $x=0$ положительное, поэтому $c>0$. \newline \null \hspace*{\fill} Ответ: $4$. 

Аналогично решаются задачи  \textbf{1461}-\textbf{1471}.

\textbf{1472.} \newline \null \hspace*{\fill} Ответ: $-1$.

\textbf{1473.} \newline \null \hspace*{\fill} Ответ: $4$.

\textbf{1474.} \newline \null \hspace*{\fill} Ответ: $5$.

\textbf{1475.} $c-$ то значение функции при $x=0$. На графике это значение разглядеть невозможно. Но можно действовать так. Будем считать, что все три коэффициента квадратного трехчлена, т.е.   и  неизвестны. Для их нахождения следует иметь координаты  трех точек, лежащих на параболе. Подставив эти координаты в уравнение функции, получим три уравнения с тремя неизвестными $a,b$ и $c$. Решив систему этих трех уравнений, найдем все коэффициенты, в том числе и $c$. Вроде бы на графике видны точки с координатами $(3;-4);\quad (4;-4)$ и $(2;2)$.

Будем считать, что именно эти точки принадлежат параболе, подставим их координаты в уравнение и получим систему уравнений:$$
	\begin{cases}
		-4=9a+3b+c
		\\
		-4=16a+4b+c
		\\
		2=4a+2b+c
	\end{cases}
	.$$

Я не знаю, умеют ли 9-классники такие системы решать. Можно исключить 
из первой и второй пары уравнений и получить систему двух уравнений с двумя неизвестными $a$ и $b$. Такие системы они решать должны. Затем находят и $c$. Я проделал эти выкладки и получил $a=3;\enspace b=-21;\enspace c=32$.  Ответ совпал с ответом у Ященко: $c=32$. \newline \null \hspace*{\fill} Ответ: $32$.

Эту идею можно попробовать применить при решении задач  \textbf{1476}-\textbf{1483}.

\textbf{1484.} Приведен классический график гиперболы $y=\frac{a}{x}$. Если $a>0$, ветви гиперболы расположены в 1 и 3 четвертях, где $x$ и $y$ имеют одинаковые знаки. В нашем случае ветви расположены во 2 и 4 четвертях, т.е. $a<0$. Ответы 1) и 3) исключаем. Чтобы выбрать между вариантами 2) и 4), подставим координаты точки $(3;-1)$, находящейся на гиперболе,  в эти уравнения и убедимся, что они удовлетворяют только уравнению 4). \newline \null \hspace*{\fill} Ответ: $4$.

Вплоть до задачи \textbf{1495} все решения основываются на приведенных выше соображениях.

\textbf{1496.} На графике явно видна точка с координатами $(-1;1)$.\newline Следовательно, $k=-1$. \newline \null \hspace*{\fill} Ответ: $-1$. 

Аналогично, находя на графике подходящую точку, можно определить значение параметра $k$ в задачах \textbf{1497}-\textbf{1499}.

Различить прямую, параболу и гиперболу несложно, задачи \newline\textbf{1500}-\textbf{1505} весьма просты.

\subsection{Графическая интерпретация уравнений, неравенств и их систем}


\textbf{506.}  Параллельные прямые имеют одинаковый угловой коэффициент. \newline \null \hspace*{\fill} Ответ: $1$. 

Задачи  \textbf{1507}-\textbf{1509} аналогичные.

\newpage \textbf{1510.} Уравнение прямой, проходящей через точки $(x_1;y_2)$ и \newline $(x_2;y_2)$, имеет вид: $$\frac{y-y_1}{y_2-y_1}=\frac{x-x_1}{x_2-x_1}.$$
Подставляя координаты точек  А и В, получим уравнение 4). \newline \null \hspace*{\fill} Ответ: $4$. 

Аналогично решаются задачи \textbf{1511}-\textbf{1513}.

\textbf{1514.} Подставляя координаты точек $С$ и $D$ в уравнения 1) и 4), можно убедиться, что  удовлетворяется только уравнение 3). \newline \null \hspace*{\fill} Ответ: $3$.

Задачи \textbf{1515}-\textbf{1517} решаются аналогично. 

\textbf{1518.} Координаты точки пересечения $(x_0;y_0)$ являются решением системы уравнений $$
			\begin{cases}
				9x-3y=3
				\\
				9x-8y=6
			\end{cases}
		  .$$
Решив эту систему, найдем, что $x_0-\frac{2}{15};\enspace y_0=-\frac{3}{5}$, т.е. точка пересечения находится в  IV четверти.          Ответ: 4.

\textbf{1519}-\textbf{1521} $-$ аналогичные задачи.

\textbf{1522.} Решением системы являются координаты точки пересечения, т.е. $(1;1)$. \newline \null \hspace*{\fill} Ответ: $(1;1)$.

Аналогично решаются задачи  \textbf{1523}-\textbf{1525}.

\textbf{1526.} Координаты точки $A$ являются решением системы $$\begin{cases}
	y=-3x-1
	\\
	y=-2x	
\end{cases}$$  и равны $(-1;2)$. \newline \null \hspace*{\fill} Ответ: $(-1;2)$. 

По аналогии можно решить задачи  \textbf{1527}-\textbf{1533}.

\textbf{1534.}  Парабола не пересекается с прямой $y=-4$, решений не имеет система 4). \newline \null \hspace*{\fill} Ответ: $4$.

\textbf{1535.}  Из графика видно, что прямая $y=2x+6$ касается параболы, а прямые $y=-4x-2$ и $x=-1$ пересекаются с ней. Значит, системы 2), 3) и 4) имеют решения (точка касания $-$ это общая точка параболы и прямой, система 3) имеет решение). Вопрос о наличии решений системы 1) не так просто решить графически. Лучше всего попробовать решить систему 1) аналитически. Нетрудно убедиться, что исключив из этой системы переменную $y$, для $x$ получим квадратное уравнение с отрицательным дискриминантом. Только тогда и становится очевидным, что система 1) решений не имеет. \newline \null \hspace*{\fill} Ответ: $1$. 

\textbf{1536.} Прямые $y=-x+1$ и $x=-1$ пересекаются с параболой, прямая $y=3x-2$ касается параболы, следовательно, системы 1), 3) и 4) решения имеют. Прямая же $y=3x-5$ параллельна прямой $y=3x-2$ и не пересекает параболу.  Система 2) решений не имеет. \newline \null \hspace*{\fill} Ответ: $2$.

\textbf{1537.}  Задача аналогична задаче \textbf{1536.} \newline \null \hspace*{\fill} Ответ: $4$.

\textbf{1538.} Задача сводится к  решению 4 систем уравнений. В каждую систему входит уравнение параболы и одно из четырех уравнений прямых. Легко выясняется тот факт, что только прямая 3) не пересекается с параболой. \newline \null \hspace*{\fill} Ответ: $3$.

Аналогично решаются задачи  \textbf{1539}-\textbf{1541}.

\textbf{1542.}  Решением системы являются координаты точек пересечения. \newline \null \hspace*{\fill} Ответ: $(-2;5),\enspace (3;0)$.

Задачи  \textbf{1543}-\textbf{1545} решаются аналогично.

\newpage \textbf{1546.} Точки $A$ и $B$ являются точками пересечения оси $Ox$ и параболы. Их абсциссы $-$ это корни квадратного уравнения $-x^2+\linebreak+9x-20=0$. Эти корни равны  $$x_{1,2}=\frac{-9\pm\sqrt{81-80}}{-2}=\frac{-9\pm1}{-2}.$$ Точке $B$ соответствует больший корень 5. \newline \null \hspace*{\fill} Ответ: $5$.

Аналогично решаются задачи \textbf{1547}-\textbf{1549} (точке $А$ в этих задачах соответствует \textbf{меньший} корень).

\textbf{1550.} Решениями системы уравнений $$\begin{cases}
	y=-x^2
	\\
	y=3x	
  \end{cases}$$ являются пары  $(-1;-3)$ и $(-2;-6)$. Точке $А$ соответствует вторая пара. \newline \null \hspace*{\fill} Ответ: $(-2;6)$.

Аналогично решаются задачи \textbf{1551}-\textbf{1553}. В задачах \textbf{1554}-\textbf{1557} в ответах надо привести координаты обеих пар решений.

В задачах \textbf{1558}-\textbf{1561} следует использовать графический метод  решения квадратных неравенств.  Методика решения  подробно обсуждалась ранее  (см., например,  задачи \textbf{1134}-\textbf{1261}). 

\textbf{1562.}  Прямая $y=-5x+1$ не пересекается с гиперболой; система 4) не имеет решений. \newline \null \hspace*{\fill} Ответ: $4$.

\textbf{1563.} \newline \null \hspace*{\fill} Ответ: $2$.     

\textbf{1564.} \newline \null \hspace*{\fill} Ответ: $2$. 

\textbf{1565.} \newline \null \hspace*{\fill} Ответ: $4$.

\textbf{1566.} \newline \null \hspace*{\fill} Ответ:  А $-$ 4, Б $-$ 1, В $-$ 2.   

\textbf{1567.} \newline \null \hspace*{\fill} Ответ:  А $-$ 42, Б $-$ 4, В $-$ 3. 

\textbf{1568.} \newline \null \hspace*{\fill} Ответ:  А $-$ 4, Б $-$ 3, В $-$ 2.   

\textbf{1569.} \newline \null \hspace*{\fill} Ответ:  А $-$ 4, Б $-$ 3, В $-$ 2.  

\textbf{1570.}  Задача сводится к  решению 4 систем уравнений. В каждую систему входит уравнение гиперболы и одно из четырех уравнений прямых. Выпишем, например, систему  $$\begin{cases}
	y=\frac{2}{x}
	\\
	y=-2x-4	
\end{cases}.$$ Исключив из системы $y$, получим уравнение $\frac{2}{x}=-2x-4$. Поскольку $x\ne0$ (эта точка  не входит в область определения гиперболы и системы уравнений), приведем уравнение к общему знаменателю и к виду $x^2+2x+1=0$. Это уравнение имеет решение (единственное), значит прямая 1) имеет одну общую точку с гиперболой.

Действуя аналогично, можно убедиться, что только в случае 4) при решении системы получается квадратное уравнение с отрицательным дискриминантом, которое действительных решений не имеет. Поэтому прямая 4) не имеет общих точек с гиперболой. \newline \null \hspace*{\fill} Ответ: $4$.

Аналогично можно решить задачи  \textbf{1571}-\textbf{1573}. В каждой из них необходимо отыскать тот случай, когда дело сводится к решению квадратного уравнения с \textbf{положительным} дискриминантом, т.к. именно в этом случае имеется 2 решения и соответствующая прямая имеет 2 общих точки с гиперболой.

\textbf{1574.} \newline \null \hspace*{\fill} Ответ:  $(-5;-1);\enspace (-1;-5)$.

\textbf{1575.} \newline \null \hspace*{\fill} Ответ:  $(-1;2);\enspace (-2;1)$.

\textbf{1576.} \newline \null \hspace*{\fill} Ответ:  $(-6;2);\enspace (2;-6)$.

\textbf{1577.} \newline \null \hspace*{\fill} Ответ:  $(-6;-1);\enspace (1;6)$.

\textbf{1578.} Система $$\begin{cases}
	y=-\frac{8}{x}
	\\
	y=x-6	
\end{cases}.$$ сводится к решению уравнения $x^2+6x+8=0$.
Его корни: $x_1=-4;\enspace x_2=-2$. Соответственно $y_1=2;\enspace y_2=4$. Точке $В$ соответствует второе решение. \newline \null \hspace*{\fill} Ответ: $(-2;4)$.

Аналогично решаются задачи  \textbf{1579}-\textbf{1585}.

\textbf{1586.} Не имеет решений система 1), т.к. прямая не пересекается с окружностью (почему-то прямой $y=-7$ на рисунке нет). \newline \null \hspace*{\fill} Ответ: $1$.

\textbf{1587.}  На рисунке опять нет одной из прямых: $y=-6$, которая касается окружности. Нет пересечений с окружностью у прямой $y=9-x$. Решений не имеет система 4). \newline \null \hspace*{\fill} Ответ: $4$.

Задачи \textbf{1588}, \textbf{1589} решаются аналогично

\textbf{1590.} \newline \null \hspace*{\fill} Ответ:  А $-$ 2, Б $-$ 1, В $-$ 4.

\textbf{1591}-\textbf{1593}  аналогичные задачи.

\textbf{1594.} Снова дело сводится к решению 4 систем уравнений. Каждая система содержит: уравнение окружности и уравнение одной из прямых. Например, запишем первую систему:$$\begin{cases}
	x^2+y^2=16
	\\
	y=3x+5	
\end{cases}.$$
Исключая из системы $y$, получим квадратное уравнение $10x^2+\linebreak+30x+9=0$. Его дискриминант положительный, система имеет 2 решения. Нам надо отыскать систему, имеющую единственное решение. Это делается только методом перебора. Можно убедиться, что только в случае 3) решение единственное. \newline \null \hspace*{\fill} Ответ: $3$.

Задачи \textbf{1595}-\textbf{1597} аналогичные.

\textbf{1598.} \newline \null \hspace*{\fill} Ответ: $(1;4);\enspace (4;1)$.

Аналогично решаются задачи  \textbf{1599}-\textbf{1601}.

\textbf{1602.} Система $$\begin{cases}
	x^2+y^2=5
	\\
	y=3x+5	
\end{cases}$$ сводится к решению уравнения $x^2+3x+2=0$. Система имеет 2 решения: $(-1;2)$ и $(-2;-1)$. Точке $А$ соответствует второе решение.\newline \null \hspace*{\fill} Ответ: $(-2;-1)$.

Аналогично решаются задачи \textbf{1603}-\textbf{1609}.

\textbf{1610.}\newline \null \hspace*{\fill} Ответ: А $-$ 1; Б $-$ 4.

Задачи \textbf{1611}-\textbf{1614}  аналогичные.
