\chapter{АЛГЕБРА}
\section{ЧИСЛА И ВЫЧИСЛЕНИЯ}
\subsection{Натуральные числа}


\textbf{1.} При делении 1580 на 9 (уголком) получим 175 и в остатке 5. Это означает, что на 9 делится без остатка число $1580-5=1575.$   
\newline \null \hspace*{\fill} Ответ: 1575.

\textbf{2.} Аналогичная задача. \newline \null \hspace*{\fill} Ответ: 616.

\textbf{3.} Аналогичная задача. \newline \null \hspace*{\fill} Ответ: 5200.

\textbf{4.} Аналогичная задача. \newline \null \hspace*{\fill} Ответ: 1176.

\textbf{5.} Аналогичная задача. \newline \null \hspace*{\fill} Ответ: 1536.

\textbf{6.} Цена деления на числовой оси равна 4. Тогда $x_A=678+4\cdot3=\newline=690$. \newline \null \hspace*{\fill} Ответ: 690. 

\textbf{7.} Аналогичная задача. $x_A=630+6\cdot4=654$. \newline \null \hspace*{\fill} Ответ: 654. 

\textbf{8.} Аналогичная задача. $x_A=332-7=325$. \newline \null \hspace*{\fill} Ответ: 325. 

\textbf{9.} Аналогичная задача. $x_A=743+2\cdot2=747$. \newline \null \hspace*{\fill} Ответ: 747. 

\textbf{10.} Аналогичная задача. $x_A=899+3=902$. \newline \null \hspace*{\fill} Ответ: 902. 

\subsection{Рациональные числа}


Задачи \textbf{11-16} $-$ на устный счет!

\textbf{11.} $8,8+5,9=14,7.$

\textbf{12.} $8,3+5,4=13,7.$

\textbf{13.} $5,6+9,7=15,3.$

\textbf{14.} $3,9-7,3=-3,4.$

\textbf{15.} $9,2-2,4=6,8.$

\textbf{16.} $3,6-4,1= -0,5.$

\textbf{17.} $8,1\cdot7,2=58,32$ (умножение столбиком, калькуляторами на ОГЭ пользоваться нельзя!).

\textbf{18.} $9,9\cdot7,1=70,29.$

\textbf{19.} $3,2\cdot6,2=19,84.$

Задачи \textbf{20-22} устно!

\textbf{20.} $$\frac{8,7}{2,9}=\frac{87}{29}=3.$$

\null \textbf{21.} $$\frac{6,5}{1,3}=5.$$

\textbf{22.} $$\frac{4,8}{0,4}=12.$$

\textbf{23.} $$\frac{1}{5}+\frac{53}{50}=0,2+1,06=1,26.$$

\textbf{24.} $$\frac{1}{5}+\frac{19}{20}=0,2+0,95=1,15.$$

\newpage \textbf{25.} $$\frac{1}{2}+\frac{33}{50}=0,5+0,66=1,16.$$

\textbf{26.} $$\frac{1}{5}-\frac{41}{50}=0,2-0,82=-0,62.$$

\textbf{27.} $$\frac{1}{5}-\frac{47}{10}=0,2-4,7=-4,5.$$

\textbf{28.} $$\frac{1}{2}-\frac{9}{10}=0,5-0,9=-0,4.$$

\textbf{29.} $$\frac{9}{5}\cdot\frac{2}{3}=\frac{6}{5}=1,2.$$
 
\textbf{30.} $$\frac{15}{4}\cdot\frac{6}{5}=\frac{9}{2}=4,5.$$

\textbf{31.} $$\frac{3}{5}\cdot\frac{25}{4}=\frac{15}{4}=3,75.$$

\textbf{32.} $$\frac{6}{5}:\frac{4}{11}=\frac{6\cdot11}{5\cdot4}=\frac{33}{10}=3,3.$$

\textbf{33.} $$\frac{12}{5}:\frac{15}{2}=\frac{12\cdot2}{5\cdot15}=\frac{8}{25}=0,32.$$

\textbf{34.} $$\frac{15}{4}:\frac{3}{7}=\frac{15\cdot7}{4\cdot3}=\frac{35}{4}=8,75.$$

\textbf{35.} $$\frac{9}{4}+\frac{8}{5}=\frac{45+32}{20}=\frac{77}{20}=3,85.$$

\textbf{36.} $$\frac{11}{4}-\frac{2}{5}=\frac{55-8}{20}=\frac{47}{20}=2,35.$$

\newpage \textbf{37.} $$\frac{1}{4}-\frac{3}{25}=\frac{25-12}{100}=0,13$$ или $$\frac{1}{4}-\frac{3}{25}=0,25-0,12=0,13.$$

\textbf{38.} $-12\cdot\left(-8,26\right)-9,4=103,2-9,4=93,8$ (умножение столбиком).

\textbf{39.} $6,8-11\cdot\left(-6,1\right)=6,8+67,1=73,9.$

\textbf{40.} $6,4-7\cdot\left(-3,3\right)=6,4+23,1=29,5.$

\textbf{41.} $$\frac{9,5+8,9}{2,3}=\frac{18,4}{2,3}=\frac{184}{23}=8.$$

\textbf{42.} $$\frac{1,3+9,2}{1,5}=\frac{10,5}{1,5}=\frac{105}{15}=7.$$

\textbf{43.} $$\frac{6,8-4,7}{1,4}=\frac{2,1}{1,4}=\frac{21}{14}=1,5.$$

\textbf{44.} $$\frac{1,5}{1+\frac{1}{5}}=\frac{1,5}{1,2}=\frac{15}{12}=1,25.$$

\textbf{45.} $$\frac{0,6}{1+\frac{1}{2}}=\frac{0,6}{1,5}=\frac{6}{15}=0,4.$$

\textbf{46.} $$\frac{1,3}{1+\frac{1}{12}}=\frac{1,3}{\frac{13}{12}}=\frac{1,3\cdot12}{13}=\frac{12}{10}=1,2.$$

\textbf{47.}  $$\frac{1}{\frac{1}{18}-\frac{1}{21}}=\frac{1}{\frac{1}{3\cdot6}-\frac{1}{3\cdot7}}=\frac{1}{\frac{7-6}{3\cdot6\cdot7}}=\frac{3\cdot6\cdot7}{1}=126.$$

\newpage \textbf{48.} $$\frac{1}{\frac{1}{35}-\frac{1}{60}}=\frac{1}{\frac{1}{5\cdot7}-\frac{1}{5\cdot12}}=\frac{1}{\frac{12-7}{5\cdot7\cdot12}}=\frac{5\cdot7\cdot12}{5}=84.$$

\textbf{49.} $$\frac{1}{\frac{1}{21}+\frac{1}{28}}=\frac{1}{\frac{1}{7\cdot3}+\frac{1}{7\cdot4}}=\frac{1}{\frac{4+3}{7\cdot3\cdot4}}=\frac{7\cdot3\cdot4}{7}=12.$$

\textbf{50.} $$\left(7\cdot10^3\right)^2\cdot\left(16\cdot10^{-4}\right)=7^2\cdot10^{3\cdot2-4}\cdot16=49\cdot16\cdot10^2=78400.$$

\textbf{51.} $$\left(9\cdot10^{-2}\right)^2\cdot\left(11\cdot10^5\right)=9^2\cdot10^{-2\cdot2+5}\cdot11=81\cdot11\cdot10^1=8910.$$

\textbf{52.} $$\left(16\cdot10^{-2}\right)^2\cdot\left(13\cdot10^4\right)=16^2\cdot10^{-2\cdot2+4}\cdot13=256\cdot13\cdot10^0=3328.$$

\textbf{53.} $$5\cdot10^{-1}+2\cdot10^{-2}+1\cdot10^{-4}=0,5+0,02+0,0001=0,5201.$$

\textbf{54.} $$2\cdot10^{-2}+8\cdot10^{-3}+5\cdot10^{-4}=0,02+0,008+0,0005=0,0285.$$

\textbf{55.} $$5\cdot10^{-1}+7\cdot10^{-3}+8\cdot10^{-4}=0,5+0,007+0,0008=0,5078.$$

\textbf{56.} $$\frac{21}{0,6\cdot2,8}=\frac{21\cdot100}{6\cdot28}=\frac{100}{2\cdot4}=12,5.$$

\newpage \textbf{57.} $$\frac{3,6\cdot4}{0,6\cdot8}=\frac{6}{2}=3.$$

\textbf{58.} $$1,4+\frac{3\cdot7,8}{2,5}=1,4+\frac{3\cdot31,2}{10}=1,4+\frac{93,6}{10}=1,4+9,36=10,76.$$

\textbf{59.} $$4\frac{3}{5}\cdot2,7=\frac{23}{5}\cdot\frac{27}{10}=\frac{621}{50}=\frac{1242}{100}=12,42.$$

\textbf{60.} $2,6\cdot6,2-0,2\cdot0,1=16,12-0,02=16,1.$

\textbf{61.}  Если заданные числа расположить в порядке возрастания, число 0,03 окажется третьим, следовательно, оно изображается точкой С. \newline \null \hspace*{\fill} Ответ: 3. 


\textbf{62.}  Заданные числа в порядке возрастания: $-0,104;\enspace -0,031;\newline -0,01;\enspace 0,1032$. Числу $-0,031$ соответствует точка В. \newline \null \hspace*{\fill} Ответ: 2. 

\textbf{63.}  Среди заданных чисел число 0,02 наибольшее, ему соответствует точка D. \newline \null \hspace*{\fill} Ответ: 4. 

\textbf{64.}  Аналогичная задача. \newline \null \hspace*{\fill} Ответ: 3. 

\textbf{65.}  Аналогичная задача. \newline \null \hspace*{\fill} Ответ: 4. 

\textbf{66.}  Учтем, что $\frac{2}{7}\approx0,286$(с точностью до 3 знака после запятой), а $\frac{3}{8}=0,375$. Заданные числа  в порядке возрастания располагаются так: $0,28;\quad\frac{2}{7}\quad;0,32;\quad\frac{3}{8}$. Точка C соответствует числу 0,32. \newline \null \hspace*{\fill} Ответ: 4. 

\textbf{67.}  Аналогичная задача  $\left(\frac{4}{3}\approx1,33;\quad\frac{6}{5}=1,2\right)$. \newline \null \hspace*{\fill} Ответ: 2. 

\textbf{68.} $\frac{5}{3}\approx1,67;\quad\frac{7}{4}=1,75$. \newline \null \hspace*{\fill} Ответ: 3.

\textbf{69.}  $\frac{2}{7}\approx0,286;\quad\frac{3}{13}\approx0,231$. \newline \null \hspace*{\fill} Ответ: 4.   

\textbf{70.} $\frac{5}{8}=0,625;\quad\frac{4}{3}\approx1,33$. \newline \null \hspace*{\fill} Ответ: 4. 

\textbf{71.} $\frac{5}{9}\approx0,56$. Остальные числа больше единицы. \newline \null \hspace*{\fill} Ответ: 1. 

\textbf{72.} Число $\frac{2}{23}<\frac{2}{20}=0,1$. Остальные числа больше 0,1. \newline \null \hspace*{\fill} Ответ: 1. 

\textbf{73.} $\frac{2}{7}\approx0,286$. Число $\frac{4}{7}$ в два раза больше, остальные числа больше 1. Отмеченная точка соответствует числу $\frac{2}{7}$. \newline \null \hspace*{\fill} Ответ: 1. 

\textbf{74.} $\frac{12}{13}$. Остальные числа меньше 0,8. \newline \null \hspace*{\fill} Ответ: 4. 

\textbf{75.} Числа расположены в возрастающем порядке. Т.к. $\frac{8}{17}<0,5$, а $\frac{9}{17}>0,5$, то правильный ответ 4. \newline \null \hspace*{\fill} Ответ: 4. 

\textbf{76.}  Т.к. $\frac{18}{17}\approx1,06$, а $\frac{17}{15}\approx1,13$, то между этими числами заключено число 1,1. \newline \null \hspace*{\fill} Ответ: 4.

\textbf{77.}  Т.к. $\frac{5}{11}\approx0,45$, а $\frac{10}{19}\approx0,53$, то между этими числами заключено число 0,5. \newline \null \hspace*{\fill} Ответ: 3. 

\textbf{78.} $\frac{5}{18}\approx0,28;\quad\frac{4}{11}\approx0,36$. \newline \null \hspace*{\fill} Ответ: 2. 

\textbf{79.} $\frac{12}{11}\approx1,09;\quad\frac{19}{17}\approx1,12$. \newline \null \hspace*{\fill} Ответ: 1. 

\textbf{80.} $\frac{15}{17}\approx0,88;\quad\frac{14}{15}\approx0,93$. \newline \null \hspace*{\fill} Ответ: 1. 

\newpage \textbf{81.} $$\left(\frac{11}{18}+\frac{2}{9}\right):\frac{5}{48}=\frac{11+4}{18}\cdot\frac{48}{5}=\frac{15\cdot48}{18\cdot5}=\frac{3\cdot8}{3}=8.$$

\textbf{82.} $$\left(\frac{11}{10}-\frac{4}{11}\right):\frac{15}{44}=\frac{121-40}{110}\cdot\frac{44}{15}=\frac{81\cdot44}{110\cdot15}=\frac{27\cdot4}{10\cdot5}=\frac{54}{25}=2,16.$$

\textbf{83.} $$\left(\frac{17}{8}-\frac{11}{20}\right):\frac{5}{46}=\frac{85-22}{40}\cdot\frac{46}{5}=\frac{63\cdot46}{40\cdot5}=$$ $$=\frac{63\cdot23}{20\cdot5}=\frac{1449}{100}=14,49.$$

\textbf{84.} $$\frac{3,9\cdot4,8}{14,4}=\frac{3,9}{3}=1,3.$$

\textbf{85.} $$\frac{2,1\cdot4,2}{9,8}=\frac{21\cdot42}{980}=\frac{3\cdot6}{20}=\frac{3\cdot3}{10}=0,9.$$

\textbf{86.} $$\frac{4,5\cdot3,2}{7,2}=\frac{45\cdot32}{720}=\frac{5\cdot32}{80}=\frac{160}{80}=2.$$

\textbf{87.} $$\frac{26}{5\cdot4}=\frac{13}{5\cdot2}=1,3.$$

\textbf{88.} $$\frac{18}{4,5\cdot2,5}=\frac{4}{2,5}=\frac{16}{10}=1,6.$$

\textbf{89.} $$\frac{12}{5\cdot4}=\frac{6}{5\cdot2}=0,6.$$

\textbf{90.} $$0,003\cdot0,3\cdot30000=3^3\cdot10^{-3}\cdot10^{-1}\cdot10^4=27\cdot10^0=27.$$

\textbf{91.} $$0,0001\cdot1\cdot100000=10^{-4}\cdot10^5=10.$$

\textbf{92.} $$0,09\cdot90\cdot90000=9^3\cdot10^{-2}\cdot10^1\cdot10^4=729\cdot10^3=729000.$$

\textbf{93.} $$\frac{0,44\cdot1,7}{4-4,6}=\frac{0,44\cdot1,7}{-0,6}=-\frac{44\cdot17}{600}=-\frac{11\cdot17}{150}=-\frac{187}{150}.$$

\textbf{94.} $$\frac{0,52\cdot6,6}{4-5,4}=\frac{0,52\cdot6,6}{-1,4}=-\frac{52\cdot66}{1400}=-\frac{13\cdot33}{175}=-\frac{429}{175}.$$

\textbf{95.} $$\frac{0,53\cdot2,3}{4-6,2}=\frac{0,53\cdot2,3}{-2,2}=-\frac{53\cdot23}{2200}=-\frac{1219}{2200}.$$

\textbf{96.} $$\left(\frac{13}{30}-\frac{11}{20}\right)\cdot\frac{9}{5}=\frac{26-33}{60}\cdot\frac{9}{5}=-\frac{7\cdot9}{60\cdot5}=-\frac{7\cdot3}{20\cdot5}=-0,21.$$

\textbf{97.} $$\left(\frac{17}{15}-\frac{1}{12}\right)\cdot\frac{20}{3}=\frac{68-5}{60}\cdot\frac{20}{3}=\frac{63\cdot20}{60\cdot3}=7.$$

\textbf{98.} $$\left(\frac{5}{33}-\frac{8}{15}\right)\cdot\frac{11}{5}=\frac{25-88}{165}\cdot\frac{11}{5}=-\frac{63\cdot11}{165\cdot5}=-\frac{63}{15\cdot5}=$$ $$=-\frac{21}{25}=-0,84.$$

 \textbf{99.} $$\left(\frac{1}{13}-2\frac{3}{4}\right)\cdot26=\left(\frac{1}{13}-\frac{11}{4}\right)\cdot26=2-\frac{11\cdot13}{2}=2-71,5=-69,5.$$

\newpage \textbf{100.} $$\left(\frac{4}{9}-3\frac{1}{15}\right)\cdot9=\left(\frac{4}{9}-\frac{46}{15}\right)\cdot9=4-\frac{46\cdot9}{15}=4-\frac{46\cdot3}{5}=$$ $$=4-27,6=-23,6.$$

\textbf{102.} Наибольшее число $9,5\cdot10^{-3}$. \newline \null \hspace*{\fill} Ответ: 4. 

\textbf{103.} Наименьшим является число $0,7\cdot10^{-5}$. \newline \null \hspace*{\fill} Ответ: 4. 

\textbf{107.} Т.к. $\frac{2}{9}\approx0,22$, то $\frac{2}{9}\in[0,2;\quad0,3]$. \newline \null \hspace*{\fill} Ответ: 2. 

\textbf{110.} Т.к. $\frac{5}{13}\approx0,38$, то $\frac{5}{13}\in[0,3;\quad0,4]$.  \newline \null \hspace*{\fill} Ответ: 2. 

\textbf{112.} $$\left(3,4\cdot10^{-2}\right)\cdot\left(5\cdot10^{-2}\right)=17\cdot10^{-4}=0,0017.$$

\textbf{115.} $$\left(8\cdot10^2\right)^2\cdot\left(3\cdot10^{-2}\right)=64\cdot3\cdot10^2=19200.$$

\textbf{118.} $$10\cdot\left(-0,1\right)^2-2,7=10\cdot\left(-0,001\right)-8\cdot0,01-2,7=$$ $$=-0,01-0,08-2,7=-2,79.$$    

\textbf{124.} $$14\cdot\left(\frac{1}{2}\right)^2+13\cdot\frac{1}{2}=\frac{14}{4}+6,5=3,5+6,5=10.$$

\textbf{128.}  $$\left(8,9\cdot10^{-4}\right)\cdot\left(4\cdot10^{-2}\right)=8,9\cdot4\cdot10^{-4-2}=35,6\cdot10^{-6}=0,0000356.$$

\textbf{130.} Т.к. $\frac{3}{4}=0,75;\quad\frac{4}{5}=0,8;\quad\frac{7}{8}=0,875$, то числу $\frac{4}{5}$ соответствует  точка D. \newline \null \hspace*{\fill} Ответ: 4. 

\textbf{132.}  $\frac{7}{11}\approx0,64;\quad\frac{4}{7}\approx0,57;\quad\frac{5}{7}\approx0,71$. Числу $\frac{7}{11}$ соответствует точка  С. \newline \null \hspace*{\fill} Ответ: 3. 

\textbf{136.} $$30-0,8\cdot\left(-10\right)^2=30-0,8\cdot100=30-80=-50.$$

\subsection{ Действительные числа}


\textbf{138.} Т.к. $7^2<56<8^2$, то $7<\sqrt{56}<8$.  Ответ: 4. 

\textbf{141.} Т.к. $7^2<60<8^2$, то $7<\sqrt{60}<8$. \newline \null \hspace*{\fill} Ответ: 4. 

\textbf{147.} Т.к. $9^2<95<10^2$, то $9<\sqrt{95}<10$. \newline \null \hspace*{\fill} Ответ: 2. 

\textbf{149.} Т.к. $\left(-3\sqrt{35}\right)^2=9\cdot35=315$ и $17^2<315<18^2$, то $17<\linebreak<3\sqrt{35}<18$, а $-18<-3\sqrt{35}<-17$. \newline \null \hspace*{\fill} Ответ: $-18;-17.$
 
\textbf{150.} Т.к. $\left(5\sqrt{27}\right)^2=25\cdot27=675$ и $25^2<315<26^2$, то $25<\linebreak<5\sqrt{27}<26$. \newline \null \hspace*{\fill} Ответ: 25;26.

\textbf{153.}  Данная задача требует более точной оценки радикалов, чем в задачах \textbf{149} и \textbf{150}.  Учтем, что $3,6^2=12,96;\enspace 3,7^2=\linebreak=13,69;\enspace 3,8^2=14,44;\enspace 3,9=15,21$. Отсюда следует, что $3,6<\linebreak<\sqrt{13}<3,7$, а $3,8<\sqrt{15}<3,9$. Из рисунка  следует, что точкой $A$ отмечено число $\sqrt{13}$. \newline \null \hspace*{\fill} Ответ: 3. 

Аналогичным  образом решаются задачи  \textbf{154} – \textbf{157}.

\textbf{158.}  Поскольку $4<\sqrt{23}<5$, то точки $N$  и $M$ исключаются. Чтобы сделать выбор между точками $Q$ и $P$, следует оценить радикал $\sqrt{23}$ более точно. Т.к. $4,5^2=20,25$, делаем вывод, что $4,5<\sqrt{23}<5$ и становится ясно, что числу $\sqrt{23}$ соответствует точка  $P$. \newline \null \hspace*{\fill} Ответ: 3.

Аналогичным  образом решаются задачи  \textbf{159} – \textbf{162}.

\textbf{163.} Из рисунка следует, что $-4<a<-3$. В этом случае только неравенство 3) является верным, остальные утверждения не верны. \newline \null \hspace*{\fill} Ответ: 3. 

\textbf{168.} Разность $x-y$ положительна, если на числовой оси точка $x$ расположена  правее точки $y$. В противном случае эта разность отрицательна. В данной задаче  положительна только разность $c-b$. \newline \null \hspace*{\fill} Ответ: 3.

Из этих соображений решаются задачи  \textbf{169} – \textbf{172}.
