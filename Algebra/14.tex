\section{ЧИСЛОВЫЕ ПОСЛЕДОВАТЕЛЬНОСТИ}
\subsection{Последовательности}


\textbf{1265.}  Для каждой  последовательности можно записать формулу для $n$–го члена: $$\text{А}: a_n=\frac{n}{n+1};\quad \text{Б}: 1+3n;\quad \text{В}: 8\cdot\left(\frac{1}{2}\right)^{n-1}.$$
Отсюда следует, что Б)$-$ арифметическая прогрессия,  В)$-$ геометрическая прогрессия, а последовательность А)$-$ не является ни арифметической, ни геометрической. \newline \null \hspace*{\fill} Ответ: $312$. 

Аналогичные задачи: \textbf{1266}-\textbf{1269}.

\textbf{1270.} $$b_2=-\frac{1}{b_1}=-\frac{1}{7};\quad b_3=-\frac{1}{b_2}=7;\quad b_4=-\frac{1}{b_4}=-\frac{1}{7}.$$ \newline \null \hspace*{\fill} Ответ: $-\frac{1}{7}$. 

\textbf{1272.} $$b_2=-\frac{3}{b_1}=\frac{1}{2};\quad b_3=-\frac{3}{b_2}=-6.$$ \newline \null \hspace*{\fill} Ответ: $-6$. 

\textbf{1275.} Для ответа на этот вопрос надо подставить в выражение для $c_n$ заданные 4 числа и решить уравнение относительно $n$. Наличие натурального корня дает положительный ответ. Легко убедиться, что правильный ответ 3). \newline \null \hspace*{\fill} Ответ: 3. 

Задачи \textbf{1276}-\textbf{1279} аналогичные.

\textbf{1280.} Задача такого же типа, только правильным ответом является число, \textbf{не} являющееся членом последовательности. Подставляя в выражение $19\cdot\frac{\left(-1\right)^n}{n}$ заданные числа, убеждаемся, что числа 1), 2) и 4) являются 21, 20 и 9 членами последовательности, а число 3) членом последовательности не является. \newline \null \hspace*{\fill} Ответ: 3.  

Таким же образом решаются задачи \textbf{1281}-\textbf{1285}.

\textbf{1286.} Мы должны выяснить, при каких $n$ $a_n>1$. Надо решить неравенство $\frac{9}{n+2}>1$. Т.к. $n-$ натуральное число, то $n+2\enspace-$ число положительное. Умножая на него неравенство (знак неравенства сохраняется), получим: $n+2<9;\quad n<7$. Таким образом, первые 6 членов последовательности меньше 1. \newline \null \hspace*{\fill} Ответ: 6. 

Задачи \textbf{1287}-\textbf{1290} аналогичные.

\subsection{Арифметическая прогрессия}


\textbf{1291.} $$a_n=a_1+d(n-1);\quad a_7=-7,7-5,3\cdot6=-39,5.$$ \newline \null \hspace*{\fill} Ответ: $-39,5$. 

\textbf{1292}-\textbf{1295} $-$ аналогичные задачи.

\textbf{1296.} $$S_n=\frac{2a_1+d(n-1)}{2}\cdot n;\quad S_{14}=\frac{-14+1,1\cdot13}{2}\cdot14=2,1.$$ \newline \null \hspace*{\fill} Ответ: $2,1$. 

\textbf{1297}-\textbf{1300} $-$ аналогичные задачи.

\textbf{1301.} Подставляя в выражение для $a_n$ $n=12$, находим $a_{12}=\linebreak=-1,5-8\cdot12=-97,5$. \newline \null \hspace*{\fill} Ответ: $-97,5$. 

\textbf{1302}-\textbf{1305}  аналогичные задачи.

\textbf{1306.} $$c_2=c_1-1=4;\quad c_3=c_2-1=3.$$ \newline \null \hspace*{\fill} Ответ: $3$.

\textbf{1307}-\textbf{1310} $-$ аналогичные задачи.

\textbf{1311.} Для заданной арифметической прогрессии $a_1=11;\linebreak d=7$. Поэтому $a_6=11+7\cdot5=46$. \newline \null \hspace*{\fill} Ответ: $46$. 

\textbf{1312}-\textbf{1320} $-$ аналогичные задачи.

\textbf{1321.} Очевидно, $d=3;\quad x=12-3=9$. \newline \null \hspace*{\fill} Ответ: $9$. 

\textbf{1322}-\textbf{1325} $-$ аналогичные задачи.

\textbf{1326.} В этой прогрессии $$a_1=2,6;\quad d=-0,3;\quad S_{17}=\frac{2\cdot2,6-0,3\cdot16}{2}\cdot17=3,4.$$ \newline \null \hspace*{\fill} Ответ: $3,4$. 

\textbf{1327}-\textbf{1330} $-$ аналогичные задачи.

\textbf{1331.} В данном случае $a_1=35;\quad d=-3;\quad a_n=35-3(n-\linebreak-1)=38-3n$. Решаем неравенство $a_n<0;\quad38-3n<0;\quad n>\linebreak>\frac{38}{3}$.  Минимальное 
значение $n$, удовлетворяющее этому условию, равно 13. Тогда $a_{13}=-1$. \newline \null \hspace*{\fill} Ответ: $-1$. 

\textbf{1332}-\textbf{1335} $-$ аналогичные задачи.

\textbf{1336.} Числа мест являются членами арифметической прогрессии, у которой $a_1=45;\quad d=2$. Для нее $a_n=45+2(n-1)=\newline=43+2n$. \newline \null \hspace*{\fill} Ответ: $43+2n$. 

\textbf{1337}-\textbf{1340} $-$ аналогичные задачи.

\textbf{1341.}  Число квадратов в  $n$-й строке можно вычислить по формуле $N_n=2+8(n-1)$. Тогда $N_{16}=2+8\cdot15=122$. \newline \null \hspace*{\fill} Ответ: $122$.

\textbf{1342}-\textbf{1345} $-$ аналогичные задачи.

\newpage \textbf{1346.} $$a_n=a_1+d(n-1);\quad d=\frac{a_n-a_1}{n-1};\quad d=\frac{7-1}{6}=1.$$ \newline \null \hspace*{\fill} Ответ: $1$.

\textbf{1347}-\textbf{1350} $-$ аналогичные задачи.

\textbf{1351.} Очевидно, $a_1=-4;\quad d=3$. Поэтому $$S_6=\frac{2\cdot(-4)+3\cdot5}{2}\cdot6=21.$$ \newline \null \hspace*{\fill} Ответ: $21$.

\textbf{1352}-\textbf{1356} $-$ аналогичные задачи.

\textbf{1357.} В данном случае $$a_1=-3;\quad d=-1,5;\quad S_6=\frac{2\cdot(-3)-1,5\cdot5}{2}\cdot6=-40,5.$$ \newline \null \hspace*{\fill} Ответ: $-40,5$.

\textbf{1358}-\textbf{1361} $-$ аналогичные задачи.

\subsection{Геометрическая прогрессия}


\textbf{1362.} $$.b_n=b_1\cdot q^{n-1};\quad b_4=140\cdot2^3=1120$$ \newline \null \hspace*{\fill} Ответ: $1120$.

\textbf{1363}-\textbf{1366} $-$ аналогичные задачи.

\textbf{1367.} $$c_1=3;\quad q=\frac{c_{n+1}}{c_n}=2;\quad c_5=c_1\cdot q^4=3\cdot2^4=48.$$ \newline \null \hspace*{\fill} Ответ: $48$. 

\textbf{1368}-\textbf{1371} $-$ аналогичные задачи.   

\textbf{1372.} $$q=\frac{-54}{18}=-3;\quad x=2\cdot q=-6.$$ \newline \null \hspace*{\fill} Ответ: $-6$.

\textbf{1373}-\textbf{1375} $-$ аналогичные задачи.

\textbf{1376.} При $n=7$ получаем: $$b_7=-6,4\cdot\left(-\frac{5}{2}\right)^7=3906,25.$$ \newline \null \hspace*{\fill} Ответ: $3906,25$.

\textbf{1377}-\textbf{1380} $-$ аналогичные задачи.

\textbf{1381.} $$q=\frac{296}{74}=4;\quad b_4=b_3\cdot q=1184\cdot4=4736.$$ \newline \null \hspace*{\fill} Ответ: $4736$.

\textbf{1382}-\textbf{1385} $-$ аналогичные задачи.

\textbf{1386.} $$a_7=a_5\cdot q^2;\quad q^2=\frac{a_7}{a_5}=\frac{1}{4};\quad q=0,5.$$ \newline \null \hspace*{\fill} Ответ: $0,5$.

\textbf{1387}-\textbf{1390} $-$ аналогичные задачи, задача \textbf{1391} повторяет \textbf{1381}.

Задачи \textbf{1391}-\textbf{1395} аналогичны задачам \textbf{1382}-\textbf{1385}.

\textbf{1396.} $$b_1=\frac{1}{9};\quad q=3;\quad S_n=b_1\cdot\frac{q^n-1}{q-1};\quad S_6=\frac{1}{9}\cdot\frac{3^6-1}{3-1}=40\frac{4}{9}.$$ \newline \null \hspace*{\fill} Ответ: $40\frac{4}{9}$. 

\textbf{1397}-\textbf{1401} $-$ аналогичные задачи.

\textbf{1402.} $$b_1=\frac{162}{3}=54;\quad q=\frac{1}{3};\quad S_4=b_1\cdot\frac{q^4-1}{q-1}=54\cdot\frac{\frac{1}{81}-1}{\frac{1}{3}-1}=80.$$ \newline \null \hspace*{\fill} Ответ: $80$.

\textbf{1403}-\textbf{1407} $-$ аналогичные задачи.

\textbf{1408.} $$b_1=-648\cdot\frac{\left(-\frac{1}{6}\right)^5-1}{-\frac{1}{6}-1}=-\frac{1111}{2}=-555,5.$$ \newline \null \hspace*{\fill} Ответ: $-555,5$.

\textbf{1409}-\textbf{1411} $-$ аналогичные задачи.
