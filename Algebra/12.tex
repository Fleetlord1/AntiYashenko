\section{АЛГЕБРАИЧЕСКИЕ ВЫРАЖЕНИЯ}
\subsection{Буквенные выражения}


\textbf{173.} Подставим  в заданное выражение $x=2$ и получим $\frac{x^3}{8}-\frac{x^2}{4}-\linebreak-5=\frac{8}{8}=\frac{4}{4}-5=-5$. \newline \null \hspace*{\fill} Ответ: $-5$. 

\textbf{174.} Аналогично, подставляя $x=-1$, найдем $0,8x^3-0,2x-4=\linebreak=-0,8+0,2-4=-4,6$. \newline \null \hspace*{\fill} Ответ: $-4,6$. 

\textbf{179.} $\left.7y^2-y+2\right|_{y=\frac{1}{7}}=\frac{7}{49}-\frac{1}{7}+2=2$. \newline \null \hspace*{\fill} Ответ: $2$.

\textbf{184.} При $x=-0,4;\quad y=-0,2;\quad z=-2,3$ получаем $2x+3y+\linebreak+z=-0,8-0,6-2,3=-3,7$. \newline \null \hspace*{\fill} Ответ: $-3,7$.

\textbf{186.} Подставляя заданные значения $a, b,c$, найдем: $\frac{a+b}{c}=\linebreak= \frac{-2,3+9,3}{-0,5}= \frac{7}{-0,5}=-14$. \newline \null \hspace*{\fill} Ответ: $-14$. 

\textbf{194.} Подставим $a=-19;\quad x=-2,9$ и получим
$\frac{a+x}{a-x}=\frac{-1,9-2,9}{-1,9+2,9}=\linebreak=-4,8$. \newline \null \hspace*{\fill} Ответ: $-4,8$. 

\textbf{197.} При $a=-9;\quad b=40$ получаем $\sqrt{a^2+b^2}=\sqrt{81+1600}=\linebreak=\sqrt{1681}=41$.\newline \null \hspace*{\fill} Ответ: $41$. 

\textbf{200.} Положим $x=-154;\quad y=-4$ и получим: 
$\sqrt{-2x+y^2}=\linebreak=\sqrt{308+16}=\sqrt{324}=18$. \newline \null \hspace*{\fill} Ответ: $18$. 

\textbf{204.} При $a=361;\quad c=16$ находим $\frac{\sqrt{a}}{\sqrt{c}-3}=\frac{\sqrt{361}}{\sqrt{16}-3}=\frac{19}{4-3}=19$. \newline \null \hspace*{\fill} Ответ: $19$. 

\textbf{208.} Если $a=0,81;\quad c=2,89$, то $\frac{\sqrt{a}}{\sqrt{c}-2}=\frac{\sqrt{0,81}}{\sqrt{2,89}-2}=\frac{0,9}{1,7-2}=\linebreak=\frac{0,9}{-0,3}=3$. \newline \null \hspace*{\fill} Ответ: $3$. 

\textbf{215.} Полагая $a=0,01;\quad b=3,24$, находим $\frac{1}{\sqrt{a}}-\sqrt{b}=\frac{1}{0,1}-1,8=\linebreak=10-1,8=8,2$. \newline \null \hspace*{\fill} Ответ: $8,2$.  

\textbf{219.} Если  $x=6,36$, то $-7\sqrt{7-x}=-7\sqrt{0,64}=-7\cdot0,8=-5,6$.\newline \null \hspace*{\fill} Ответ: $-5,6$. 

\textbf{224.} Если $x=1,59$, то $-3\sqrt{10-x}=-3\sqrt{8,41}=-3\cdot2,9=-8,7$. \newline \null \hspace*{\fill} Ответ: $-8,7$. 

\subsection{Многочлены}


\textbf{226.} Если у слагаемых в каждой  скобке  изменить знаки на противоположные,  то выражение не изменится. \newline \null \hspace*{\fill} Ответ: $3$. 

\textbf{230.} Если из некоторого выражения выносится общий множитель за скобки, то в скобках все слагаемые необходимо на этот множитель разделить. \newline \null \hspace*{\fill} Ответ: $2$. 

Аналогично решаются задачи \textbf{231-233}.

\textbf{234.} Разложение на множители  квадратного трехчлена имеет вид:
$$ax^2+bx+c=a\left(x-x_1\right)\left(x-x_2\right),$$ где $$x_{1,2}=\frac{-b\pm\sqrt{b^2-4ac}}{2a}-$$ $-$ его корни. В нашем случае $8x^2+8x-16=8(x^2+x-2)=$\newline$=8\left(x+2\right)\left(x-1\right)$. Корни квадратного трехчлена $x^2+x+2$ найдены так:
$$x_{1,2}=\frac{-1\pm\sqrt{1+8}}{2}=\frac{-1\pm3}{2}, x_1=-2;\quad x_2=1.$$ \newline \null \hspace*{\fill} Ответ: $x-1$. 

Аналогичным образом можно решить задачи  \textbf{235-237}. В задаче \textbf{237} проще воспользоваться формулой $x^2-a^2=\left(x-a\right)\left(x+a\right)$.

\textbf{238.} Квадратный трехчлен $ax^2+bx+c$ не раскладывается на множители, если его дискриминант $b^2-4ac<0$. В нашей задаче это имеет место только в случае 2): $4^2-4\cdot7<0$. \newline \null \hspace*{\fill} Ответ: $2$. 

Аналогично решаются задачи  \textbf{239-241}.

\textbf{242.}  Идея решения задачи весьма проста: надо раскрыть выражение в левой части и сравнить его с выражением в правой части; тождественное равенство имеет место только в случае 3). \newline \null \hspace*{\fill} Ответ: $3$. 

Так же решатся задача  \textbf{243.}

\textbf{244.}  Чтобы преобразовать выражение в многочлен, необходимо раскрыть все скобки и привести подобные члены. Иногда для этого следует применить формулы сокращенного умножения. В данной задаче:  
$$\left(a-b\right)^2\left(a+b\right)=\left(a-b\right)\left(a^2-b^2\right)=a^3-a^2b-ab^2+b^3.$$
Неписаное правило: в слагаемых буквы расставляются по алфавиту, а степени первой буквы$-$ в убывающем порядке. \newline \null \hspace*{\fill} Ответ: $a^3-a^2b-ab^2+b^3$.  

Задачи \textbf{245-247} решаются аналогично.

\textbf{248.}  
$$\left(b+4\right)^2-2b\left(5b+4\right)=b^2+8b+16-10b^2-8b=$$ $$=-9b^2+16=\left(4-3b\right)\left(4+3b\right).$$ \newline \null \hspace*{\fill} Ответ: $(4-3b)(4+3b)$. 

Задачи \textbf{249-251} решаются аналогично.

\textbf{252.} $$7c\left(4c+2\right)-\left(7+c\right)=28c^2+14c-49-14c-c^2=27c^2-49.$$ \newline \null \hspace*{\fill}Ответ: $27c^2-49$.

Аналогично решаются задачи \textbf{253-255}.

\textbf{256.} $$12a-2\left(a+3\right)^2=12a-2\left(a^2+6a+9\right)=-2a^2-18.$$
\newline \null \hspace*{\fill} Ответ: $-2a^2-18$. 

\textbf{260.} $$\left(3x-8y\right)^2+6x\left(9x+8y\right)=9x^2-48xy+64y^2+54x^2+48xy=$$ $$=63x^2+64y^2.$$ \newline \null \hspace*{\fill} Ответ: $63x^2+64y^2$. 

\textbf{272.}  Как правило, в таких задачах не сразу подставляют числовые значения, а предварительно заданное выражение упрощают: $$4a-2\left(a+1\right)^2=4a-2a^2-4a-2=-2a^2-2.$$ Если $a=\sqrt{5}$, то $-2a^2-2=-2\cdot5-2=-12.$ \newline \null \hspace*{\fill} Ответ: $-12$. 

\textbf{276.} $\left(5b+1\right)^2-10b\left(2b+1\right)=25b^2+10b+1-20b^2-10b=5b^2+1.$ При $b=\sqrt{29}$, получаем: $5b^2+1=5\cdot29+1=146$. \newline \null \hspace*{\fill} Ответ: $146$.

Задачи \textbf{273-275, 277-283} решаются аналогично. 

\textbf{284.} $\left(x+y\right)^2+2x(3x-y)=x^2+2xy+y^2+6x^2-2xy=7x^2+y^2$. Подставляя теперь $x=1;\quad y=\sqrt{2}$, получаем: $7x^2+y^2=7\cdot1+2=\linebreak=9$. \newline \null \hspace*{\fill} Ответ: $9$. 

Задачи  \textbf{285-294}  решаются аналогично. 

\textbf{295.} $-10ab+5\left(a+b\right)^2=5\left(a^2+b^2\right)=5\left(12+5\right)=85$.  \null \hspace*{\fill} Ответ: $9$. 

\newpage \subsection{Алгебраические дроби}


\textbf{296.} Ответ: 3. \textbf{297.} Ответ: 2. \textbf{298.} Ответ: 3.

\textbf{299.} Сократить дробь$-$ это значит: разложить числитель и знаменатель на множители и сократить одинаковые множители. Иногда приходится использовать формулы сокращенного умножения.
$$\frac{2ab}{ab+3a^2}=\frac{2ab}{a\left(b+3a\right)}=\frac{2b}{b+3a}.$$ \newline \null \hspace*{\fill} Ответ: $\frac{2b}{b+3a}$. 

Задачи \textbf{300-304} решаются аналогично. 

\textbf{305.} $$\frac{a}{ab-2b^2}:\frac{4a^2}{a^2-4ab+4b^2}=\frac{a\cdot\left(a-2b\right)^2}{b\left(a-2b\right)\cdot4a^2=\frac{a-2b}{4ab}}.$$

Чтобы избежать длинных выкладок и переписываний, надо стараться одновременно упрощать (раскладывать на множители, применять формулы сокращенного умножения, деление на дробь заменять умножением на обратную дробь и т.д.) во всех фрагментах примера. \newline \null \hspace*{\fill} Ответ: $\frac{a-2b}{4ab}$. 

\textbf{311.}  $$\frac{10a^2-b^2}{6a^2}\cdot\frac{a}{20a-2b}=\frac{\left(10a-b\right)\left(10a+b\right)a}{12a^2\left(10a-b\right)}=\frac{10a+b}{12a}.$$  \null \hspace*{\fill} Ответ: $\frac{10a+b}{12a}$.

\textbf{317.} $$\frac{1}{6x}-\frac{6x+y}{6xy}=\frac{y-6x-y}{6xy}=\frac{-6x}{6xy}=-\frac{1}{y}.$$ 

\newpage Можно и так: $$\frac{1}{6x}-\frac{6x+y}{6xy}=\frac{1}{6x}-\frac{1}{y}=\frac{1}{6x}=-\frac{1}{y}.$$  \null \hspace*{\fill} Ответ: $-\frac{1}{y}$.

\textbf{320.} $$\frac{2a}{a^2-25b^2}-\frac{2}{a+5b}=\frac{2a}{\left(a-5b\right)\left(a+5b\right)}-\frac{2}{a+5b}=$$ $$=\frac{2a-2\left(a-5b\right)}{\left(a-5b\right)\left(a+5b\right)}=\frac{10b}{\left(a-5b\right)\left(a+5b\right)}.$$ \newline \null \hspace*{\fill} Ответ: $\frac{10b}{\left(a-5b\right)\left(a+5b\right)}$.

\textbf{Замечание.} Я по-прежнему предпочитаю такой ответ ответу в задачнике. Впрочем, задачи из данного задачника носят тренировочный характер. На экзамене задачи будут несколько другие, и вряд ли появятся сомнения, в каком виде записывать ответ. 

\textbf{326.} $$\frac{2}{a}-\frac{-2a^2+9b^2}{ab}-\frac{2a}{b}=\frac{2b+2a^2-9b^2-2a^2}{ab}=\frac{2b-9b^2}{ab}=\frac{2-9b}{a}.$$ \newline \null \hspace*{\fill} Ответ: $\frac{2-9b}{a}$.

\textbf{333.} $$\left(\frac{9y}{x}-\frac{49x}{y}\right):\left(3y-7x\right)=\frac{9y^2-49x^2}{xy\left(3y-7x\right)}=\frac{\left(3y-7x\right)\left(3y+7x\right)}{xy\left(3y-7x\right)}=$$ $$=\frac{3y+7x}{xy}.$$ \newline \null \hspace*{\fill} Ответ: $\frac{3y+7x}{xy}$.

\newpage \textbf{338.} $$\frac{b}{18b-81}:\frac{4b^2}{4b^2-81}=\frac{b\left(2b-9\right)\left(2b+9\right)}{9\left(2b-9\right)\cdot4b^2}=\frac{2b+9}{36b}.$$ \newline \null \hspace*{\fill} Ответ: $\frac{2b+9}{36b}$.

\textbf{343.} $$\frac{3a}{9a^2-15ab}-\frac{5b}{9a^2-25b^2}=\frac{3a}{3a\left(3a-5b\right)}-\frac{5b}{\left(3a-5b\right)\left(3a+5b\right)}=$$ $$=\frac{1}{3a-5b}-\frac{5b}{\left(3a-5b\right)\left(3a+5b\right)}=\frac{3a+5b-5b}{\left(3a-5b\right)\left(3a+5b\right)}=$$ $$=\frac{3a}{\left(3a-5b\right)\left(3a+5b\right)}.$$ \newline \null \hspace*{\fill} Ответ: $\frac{3a}{\left(3a-5b\right)\left(3a+5b\right)}$.

\textbf{346.} $$\frac{81a^2-b^2}{\left(9a-b\right)^2}=\frac{\left(9a-b\right)\left(9a+b\right)}{\left(9a-b\right)^2}=\frac{9a+b}{9a-b}.$$ \newline \null \hspace*{\fill} Ответ: $\frac{9a+b}{9a-b}$. 

\textbf{347.} $$\frac{x^2-25}{x^2-3x-10}=\frac{\left(x-5\right)\left(x+5\right)}{\left(x-5\right)\left(x+2\right)}=\frac{x+5}{x+2}.$$ \newline \null \hspace*{\fill} Ответ: $\frac{x+5}{x+2}$. 

\textbf{351.} $$\frac{\left(2x+3y\right)^2-\left(2x-3y\right)^2}{x}=\frac{4x^2+12xy+9y^2-4x^2+12xy-9y^2}{x}=$$ $$=\frac{24xy}{x}=24y.$$ \newline \null \hspace*{\fill} Ответ: $24y$. 

\newpage \textbf{355.} $$\frac{n^3+n^2}{n^2-1}=\frac{n^2\left(n+1\right)}{\left(n-1\right)\left(n+1\right)}=\frac{n^2}{n-1}.$$ \newline \null \hspace*{\fill} Ответ: $\frac{n^2}{n-1}$. 


\textbf{356.} $$\frac{b}{a-b}\cdot\left(\frac{1}{a}-\frac{1}{b}\right)=\frac{b\left(b-a\right)}{\left(a-b\right)ab}=-\frac{1}{a}.$$ \newline \null \hspace*{\fill} Ответ: $-\frac{1}{a}$. 

\textbf{362.} $$\left(\frac{x^3-8}{x+2}\right)\cdot\left(\frac{x^2+4x+4}{x^2+2x+4}\right)=\frac{\left(x-2\right)\left(x^2+2x+4\right)\left(x+2\right)^2}{\left(x+2\right)\left(x^2+2x+4\right)}=$$ $$=\left(x-2\right)\left(x+2\right)=x^2-4.$$ \newline \null \hspace*{\fill} Ответ: $x^2-4$. 

\textbf{366.} $$\frac{8b^3+12b^2+6b+1}{b}:\left(\frac{1}{b}+2\right)=\frac{\left(2b+1\right)^3}{b}\cdot\frac{b}{1+2b}=\left(2b+1\right)^2.$$ \newline \null \hspace*{\fill} Ответ: $\left(2b+1\right)^2$.

\textbf{368.} Если $x=0,3$, то $$\frac{1}{x}-\frac{2}{5x}=\frac{5-2}{5x}=\frac{3}{5x}=\frac{3}{5\cdot0,3}=2.$$ \newline \null \hspace*{\fill} Ответ: $2$. 

\textbf{372.}  Упростим выражение и подставим $a=-3$: $$\frac{28}{4a-a^2}-\frac{7}{a}=\frac{28}{a\left(4-a\right)}-\frac{7}{a}=\frac{28-7\left(4-a\right)}{a\left(4-a\right)}=\frac{7a}{a\left(4-a\right)}=$$ $$=\frac{7}{4-a}=\frac{7}{4+3}=1.$$ \newline \null \hspace*{\fill} Ответ: $1$.

\textbf{377.} $$9b+\frac{5a-9b^2}{b}=9b+\frac{5a}{b}-9b=\frac{5a}{b}=\frac{5\cdot9}{18}=2,5.$$ \newline \null \hspace*{\fill} Ответ: $2,5$.

\textbf{381.} Сначала упростим: $$\frac{a+x}{a}:\frac{ax+x^2}{a^2}=\frac{\left(a+x\right)\cdot a^2}{ax\left(a+x\right)}.$$ При $a=56,x=40$ получим $\frac{a}{x}=\frac{56}{40}=1,4$. \newline \null \hspace*{\fill} Ответ: $1,4$.

\textbf{384.} $$\frac{xy+y^2}{8x}\cdot\frac{4x}{x+y}=\frac{y\left(x+y\right)\cdot4x}{8x\left(x+y\right)}=\frac{y}{2}=-2,6.$$ \newline \null \hspace*{\fill} Ответ: $-2,6$. 

\textbf{392.} $$\frac{5b}{a-b}\cdot\frac{a^2-ab}{25b}=\frac{5ba\left(a-b\right)}{\left(a-b\right)\cdot25b}=\frac{a}{5}=\frac{36}{5}=7,2.$$ \newline \null \hspace*{\fill} Ответ: $7,2$. 

\textbf{395.} $$\frac{9ab}{a+9b}\cdot\left(\frac{a}{9b}-\frac{9b}{a}\right)=\frac{9ab\cdot\left(a^2-81b^2\right)}{\left(a+9b\right)\cdot9ab}=\frac{\left(a-9b\right)\left(a+9b\right)}{a+9b}=a-9b.$$
Полагая $a=9\sqrt{8}+6;\quad b=\sqrt{8}-9$, получаем: $a-9b=87.$ \newline \null \hspace*{\fill} Ответ: $87$. 

\textbf{398.} $$\frac{a^2-9b^2}{3ab}:\left(\frac{1}{3b}-\frac{1}{a}\right)=\frac{\left(a-3b\right)\left(a+3b\right)\cdot3ab}{3ab\cdot\left(a-3b\right)}=a+3b.$$ Если $a=2\frac{2}{17}=\frac{36}{17};\quad b=9\frac{5}{17}=\frac{158}{17}$, то $a+3b=\frac{36+3\cdot158}{17}=30.$ \newline \null \hspace*{\fill} Ответ: $30$. 

\newpage \textbf{401.} $$\frac{9a}{8c}-\frac{81a^2+64c^2}{72ac}+\frac{8c-81a}{9a}=\frac{81a^2-81a^2-64c^2+64c^2-648ac}{72ac}=$$ $$=\frac{-648ac}{72ac}=-9.$$ \newline \null \hspace*{\fill} Ответ: $-9$. 

Далее задачи (вплоть до \textbf{431})  аналогичные и однотипные, смысла решать их все нет никакого, достаточно решить 3-4.

\textbf{432.} $$\frac{n^3-\sqrt{2}n^2}{n^2-2}=\frac{n^2\left(n-\sqrt{2}\right)}{\left(n-\sqrt2\right)\left(n+\sqrt{2}\right)}=\frac{n^2}{n+\sqrt{2}}=\frac{8}{3\sqrt{2}}=\frac{4\sqrt{2}}{3}.$$ \newline \null \hspace*{\fill} Ответ: $\frac{4\sqrt{2}}{3}$. 

\textbf{436.} $$\frac{a}{35a-25}:\frac{a^2}{49a^2-70a+25}=\frac{a\left(7a-5\right)^2}{5\left(7a-5\right)\cdot a^2}=\frac{7a-5}{5a}=\frac{7}{5}-\frac{1}{a}=$$ $$=\frac{7}{5}-\frac{8}{5}=-0,2.$$ \newline \null \hspace*{\fill} Ответ: $-0,2$. 

\textbf{440.} $$\frac{b^2}{36a^2-b^2}:\frac{b}{6a-b}=\frac{b^2\left(6a-b\right)}{\left(6a-b\right)\left(6a+b\right)\cdot b}=\frac{b}{6a+b}=\frac{q}{1+\frac{6a}{b}}=$$ $$=\frac{1}{1+9}=0,1.$$ \newline \null \hspace*{\fill} Ответ: $0,1$. 

\newpage\textbf{446.} $$\frac{a^2-64}{a^2}\cdot\frac{a}{a+8}=\frac{\left(a-8\right)\left(a+8\right)\cdot a}{a^2\left(a+8\right)}=\frac{a-8}{a}=1-\frac{8}{a}=$$ $$=1-8\cdot23=-183.$$ \newline \null \hspace*{\fill} Ответ: $-183$. 

\textbf{452.} $$\left(u+\upsilon+\frac{\upsilon^2}{u}\right):\left(1+\frac{\upsilon}{u}\right)=\frac{\left(u+\upsilon\right)^2}{u}\cdot\frac{u}{u+\upsilon}=u+\upsilon=14.$$ \newline \null \hspace*{\fill} Ответ: $14$. 

\textbf{456.} $$\left(a^2-3a-\frac{1}{a}+3\right)\cdot\frac{1}{a^2-1}\cdot\left(a^2+a\right)=\frac{\left(a-1\right)^3\cdot a\left(a+1\right)}{a\left(a-1\right)\left(a+1\right)}=$$ $$=\left(a-1\right)^2=1,5^2=2,25.$$ \newline \null \hspace*{\fill} Ответ: $2,25$.

\textbf{459.} $$\left(a^2+12a+\frac{64}{a}+48\right)\cdot\frac{1}{a^2-16}\cdot\left(a^2-4a\right)=$$ $$=\frac{\left(a+4\right)^3}{a}\cdot\frac{1}{\left(a-4\right)\left(a+4\right)}\cdot a\left(a-4\right)=\left(a+4\right)^2=$$ $$=\left(-5,5+4\right)^2=2,25.$$ \newline \null \hspace*{\fill} Ответ: $2,25$.

\textbf{461.} $$\left(\frac{x}{y}+\frac{9y}{x}-6\right)\cdot\frac{1}{\left(x-3y\right)^2}=\frac{\left(x-3y\right)^2}{xy\left(x-3y\right)^2}=\frac{1}{xy}=\frac{1}{\sqrt{5\cdot0,2}}=1.$$ \newline \null \hspace*{\fill} Ответ: $1$. 

\textbf{462.} $$\left(\frac{49x}{y}+\frac{9y}{x}-42\right)\cdot\frac{1}{\left(7x-3y\right)^2}=\frac{\left(7x-3y\right)^2}{xy}\cdot\frac{1}{\left(7x-3y\right)}=$$ $$=\frac{1}{xy}=\frac{1}{\sqrt{15\cdot\frac{5}{3}}}=\frac{1}{5}=0,2.$$ \newline \null \hspace*{\fill} Ответ: $0,2$.

\textbf{466.} $$\left(\frac{9a}{b}-\frac{b}{a}\right):\left(1-\frac{3a}{b}\right)=\frac{9a^2-b^2}{ab}\cdot\frac{b}{b-3a}=-\frac{\left(3a-b\right)\left(3a+b\right)}{a\left(3a-b\right)}=$$ $$=-\frac{3a+b}{a}=-3-\frac{b}{a}=-3+\frac{1}{2}=-2,5.$$ \newline \null \hspace*{\fill} Ответ: $-2,5$.

\textbf{472.} $$\frac{5a}{25a^2-15ab}-\frac{3b}{25a^2-9b^2}=\frac{1}{5a-3b}-\frac{3b}{\left(5a-3b\right)\left(5a+3b\right)}=$$ $$=\frac{5a+3b-3b}{\left(5a-3b\right)\left(5a+3b\right)}=\frac{5a}{25a^2-9b^2}=\frac{-5}{25-27}=2,5.$$ \newline \null \hspace*{\fill} Ответ: $2,5$. 

\textbf{476.} $$\frac{\left(4x+2y\right)^2-\left(4x-2y\right)^2}{x}=$$$$=\frac{\left(4x+2y-4x+2y\right)\left(4x+2y+4x-2y\right)}{x}=\frac{4y\cdot8x}{x}=32y=160.$$ \newline \null \hspace*{\fill} Ответ: $160$. 

\textbf{478.} $$\frac{\left(x+2y\right)^2-\left(x-2y\right)^2}{x}=\frac{\left(x+2y-4+2y\right)\left(x+2y+x-2y\right)}{x}=$$ $$=\frac{4y\cdot2x}{x}=8y=-56.$$ \newline \null \hspace*{\fill} Ответ: $-56$. 

\subsection{Степени с целыми показателями и их свойства}


\textbf{479.}  Ответ: 1.  \textbf{482.}  Ответ: 2.  \textbf{485.}  Ответ: 4.  \textbf{488.}  Ответ: 4.  \textbf{492.}  Ответ: 4. \textbf{496.}  Ответ: 1. \textbf{499.}  Ответ: $x^{-8}$.   \textbf{502.}  Ответ: 1.  \textbf{506.} Ответ: 3. \textbf{510.}  Ответ: 2.  \textbf{513.}  Ответ: 3.

\subsection{Квадратный корень и его свойства}


\textbf{514.} Если $a>b>0$, то и $\sqrt{a}>\sqrt{b}$. Поэтому, чтобы сравнить несколько квадратных корней, достаточно сравнить их квадраты или подкоренные выражения.  В данной задаче: $$\left(\sqrt{55}\right)^2=55;\quad \left(\sqrt2{7}\right)^2=4\cdot7=28;\quad 7^2=49;$$$$ \left(2\sqrt{13}\right)^2=4\cdot13=52$$ \newline \null \hspace*{\fill} Ответ: $1$. 

\textbf{517.} Иногда для сравнения чисел полезно составить их разность: если эта разность положительна, то уменьшаемое больше вычитаемого. Так, разность $5\sqrt{3}-\left(2\sqrt{3}+3\sqrt{2}\right)=3\sqrt{3}-3\sqrt{2}>0$, т.к. $\sqrt{3}>\sqrt{2}$. Случай 4) исключаем. Теперь можно сравнить числа из первых трех вариантов и убедиться, что наибольшее число $5\sqrt{3}$. \newline \null \hspace*{\fill} Ответ: $2$. 

\textbf{521.}  Сравним  предлагаемые  числа. Их квадраты  равны соответственно $21;\quad 54;\quad 36;\quad \frac{102}{3}=34$. Следовательно, правильный ответ: 2. \newline \null \hspace*{\fill} Ответ: $2$. 

\textbf{526.} Квадраты чисел: $49;\quad 50;\quad 48$. \newline \null \hspace*{\fill} Ответ: $4$. 

\textbf{528.} Квадраты чисел: $90;\quad90,25;\quad89$. \newline \null \hspace*{\fill} Ответ: $3$. 

\textbf{529.} $\frac{\sqrt{343}}{\sqrt{7}}=\sqrt{\frac{343}{7}}=\sqrt{49}=7$. \newline \null \hspace*{\fill} Ответ: $3$. 

\textbf{532.} $\sqrt{18\cdot80}\cdot\sqrt{30}=\sqrt{18\cdot80\cdot30}=\sqrt{9\cdot2\cdot8\cdot10\cdot3\cdot10}=4\cdot3\cdot\linebreak\cdot10\sqrt{3}=120\sqrt{3}$. \newline \null \hspace*{\fill} Ответ: $4$. 

\textbf{535.} $$\frac{\left(2\sqrt{6}\right)^2}{36}=\frac{4\cdot6}{36}=\frac{2}{3}.$$ \newline \null \hspace*{\fill} Ответ: $1$.

\textbf{540.} Для \textbf{быстрого} решения  использован тот факт, что в вариантах ответов имеется число . Поэтому делим 432 на 3, получаем 144, $\sqrt{144}=12$.    

Другой вариант решения: используем признак делимости целого числа на $9- $сумма цифр кратна $9$. Тогда $\sqrt{432}=\sqrt{9\cdot48}=\linebreak=\sqrt{9\cdot16\cdot3}=12\sqrt{3}$.
 
На самый худой конец необходимо число $432$ разложить на простейшие множители и извлечь корень. Но это самый длинный путь решения. Надо приучать себя к приемам, основанным на устном счете (см. решение  задачи  \textbf{532} и два первых решения задачи  \textbf{540}). \newline \null \hspace*{\fill} Ответ: $3$. 

\textbf{543.} Выражения $a\pm\sqrt{b}$ или $\sqrt{a}\pm\sqrt{b}$ называются сопряженными. Их произведение не содержит радикалов. На использовании этого факта основано освобождение от иррациональности в знаменателе. Например, в данной задаче: $$\frac{1}{3-\sqrt{7}}=\frac{3+\sqrt{7}}{\left(3-\sqrt{7}\right)\left(3+\sqrt{7}\right)}=\frac{3+\sqrt{7}}{9-7}=\frac{3+\sqrt{7}}{2}.$$ \newline \null \hspace*{\fill} Ответ: $3$.

\textbf{546.} Поскольку задано иррациональное число, варианты ответов \textbf{1} и \textbf{3} исключаются сразу. Чтобы привести заданное число к виду \textbf{2} или \textbf{4}, подведем число $4$ под знак квадратного корня, возведя его в квадрат: $$4\sqrt{6}=\sqrt{16\cdot6}=\sqrt{96}.$$ \newline \null \hspace*{\fill} Ответ: $2$. 

\textbf{548.} $\sqrt{3^6}=3^3=27$. \newline \null \hspace*{\fill} Ответ: $2$. 

\textbf{549.} $\sqrt{4^5}=\left(\sqrt{4}\right)^5=2^5=32$. \newline \null \hspace*{\fill} Ответ: $4$. 

\textbf{550.} $\sqrt{60}-\sqrt{15}=\sqrt{4\cdot15}-\sqrt{15}=2\sqrt{15}-\sqrt{15}=\sqrt{15}$. \newline \null \hspace*{\fill} Ответ: $2$. 

\textbf{555.} $\sqrt{12}+\sqrt{27}=2\sqrt{3}+3\sqrt{3}=5\sqrt{3}$. В списке ответов нет правильного. Либо в условии, либо в ответах опечатка.

\textbf{558.} $\left(\sqrt{19}-4\right)\left(\sqrt{19}+4\right)=19-16=3$. \newline \null \hspace*{\fill} Ответ: $4$. 

\textbf{560.} $$\frac{6}{\left(2\sqrt{3}\right)^2}=\frac{6}{4\cdot3}=0,5.$$ \newline \null \hspace*{\fill} Ответ: $2$. 

\textbf{563.} $$\frac{\sqrt8\cdot\sqrt{675}}{\sqrt{60}}=\sqrt{\frac{8\cdot25\cdot27}{4\cdot15}}=\sqrt{2\cdot5\cdot9}=3\sqrt{10}.$$ \newline \null \hspace*{\fill} Ответ: $3$.

\textbf{567.} $$2\sqrt6\cdot2\sqrt2\cdot8\sqrt3=16\sqrt{36}=16\cdot6=96.$$ \newline \null \hspace*{\fill} Ответ: $1$.

\textbf{570.} $$\sqrt{1,28}\cdot\frac{1}{\sqrt8}=\sqrt{\frac{1,28}{8}}=\sqrt{0,16}=0,4.$$ \newline \null \hspace*{\fill} Ответ: $0,4$.

\textbf{573.} $$\left(\sqrt{40}+4\right)^2=40+8\sqrt{40}+16=56+8\sqrt{40}.$$ \newline \null \hspace*{\fill} Ответ: $3$.

\textbf{574.} $$\sqrt{810000}=900;\quad \sqrt{810}=9\sqrt{10};\quad \sqrt{81}=9.$$ \newline \null \hspace*{\fill} Ответ: $2$. 

\textbf{577.} $$\sqrt{121}=11;\quad \sqrt{0,36}=0,6;\quad \sqrt{7\frac{8}{17}}=\sqrt{\frac{127}{17}}.$$ \newline \null \hspace*{\fill} \newline \null \hspace*{\fill} Ответ: $3$. 

\textbf{580.} От радикалов можно избавиться только в выражении $\sqrt{20}\cdot$\newline$\cdot\sqrt5=\sqrt{100}=10$. \newline \null \hspace*{\fill} Ответ: $1$.

\textbf{582.} Только одно выражение $\frac{\sqrt2}{\sqrt{20}}=\frac{1}{\sqrt{10}}$ является иррациональным. В остальных выражениях от радикалов можно освободиться. \newline \null \hspace*{\fill} Ответ: $2$.